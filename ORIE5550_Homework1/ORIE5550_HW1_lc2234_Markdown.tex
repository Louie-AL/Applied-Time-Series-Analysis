% Options for packages loaded elsewhere
\PassOptionsToPackage{unicode}{hyperref}
\PassOptionsToPackage{hyphens}{url}
%
\documentclass[
]{article}
\usepackage{amsmath,amssymb}
\usepackage{iftex}
\ifPDFTeX
  \usepackage[T1]{fontenc}
  \usepackage[utf8]{inputenc}
  \usepackage{textcomp} % provide euro and other symbols
\else % if luatex or xetex
  \usepackage{unicode-math} % this also loads fontspec
  \defaultfontfeatures{Scale=MatchLowercase}
  \defaultfontfeatures[\rmfamily]{Ligatures=TeX,Scale=1}
\fi
\usepackage{lmodern}
\ifPDFTeX\else
  % xetex/luatex font selection
\fi
% Use upquote if available, for straight quotes in verbatim environments
\IfFileExists{upquote.sty}{\usepackage{upquote}}{}
\IfFileExists{microtype.sty}{% use microtype if available
  \usepackage[]{microtype}
  \UseMicrotypeSet[protrusion]{basicmath} % disable protrusion for tt fonts
}{}
\makeatletter
\@ifundefined{KOMAClassName}{% if non-KOMA class
  \IfFileExists{parskip.sty}{%
    \usepackage{parskip}
  }{% else
    \setlength{\parindent}{0pt}
    \setlength{\parskip}{6pt plus 2pt minus 1pt}}
}{% if KOMA class
  \KOMAoptions{parskip=half}}
\makeatother
\usepackage{xcolor}
\usepackage[margin=1in]{geometry}
\usepackage{color}
\usepackage{fancyvrb}
\newcommand{\VerbBar}{|}
\newcommand{\VERB}{\Verb[commandchars=\\\{\}]}
\DefineVerbatimEnvironment{Highlighting}{Verbatim}{commandchars=\\\{\}}
% Add ',fontsize=\small' for more characters per line
\usepackage{framed}
\definecolor{shadecolor}{RGB}{248,248,248}
\newenvironment{Shaded}{\begin{snugshade}}{\end{snugshade}}
\newcommand{\AlertTok}[1]{\textcolor[rgb]{0.94,0.16,0.16}{#1}}
\newcommand{\AnnotationTok}[1]{\textcolor[rgb]{0.56,0.35,0.01}{\textbf{\textit{#1}}}}
\newcommand{\AttributeTok}[1]{\textcolor[rgb]{0.13,0.29,0.53}{#1}}
\newcommand{\BaseNTok}[1]{\textcolor[rgb]{0.00,0.00,0.81}{#1}}
\newcommand{\BuiltInTok}[1]{#1}
\newcommand{\CharTok}[1]{\textcolor[rgb]{0.31,0.60,0.02}{#1}}
\newcommand{\CommentTok}[1]{\textcolor[rgb]{0.56,0.35,0.01}{\textit{#1}}}
\newcommand{\CommentVarTok}[1]{\textcolor[rgb]{0.56,0.35,0.01}{\textbf{\textit{#1}}}}
\newcommand{\ConstantTok}[1]{\textcolor[rgb]{0.56,0.35,0.01}{#1}}
\newcommand{\ControlFlowTok}[1]{\textcolor[rgb]{0.13,0.29,0.53}{\textbf{#1}}}
\newcommand{\DataTypeTok}[1]{\textcolor[rgb]{0.13,0.29,0.53}{#1}}
\newcommand{\DecValTok}[1]{\textcolor[rgb]{0.00,0.00,0.81}{#1}}
\newcommand{\DocumentationTok}[1]{\textcolor[rgb]{0.56,0.35,0.01}{\textbf{\textit{#1}}}}
\newcommand{\ErrorTok}[1]{\textcolor[rgb]{0.64,0.00,0.00}{\textbf{#1}}}
\newcommand{\ExtensionTok}[1]{#1}
\newcommand{\FloatTok}[1]{\textcolor[rgb]{0.00,0.00,0.81}{#1}}
\newcommand{\FunctionTok}[1]{\textcolor[rgb]{0.13,0.29,0.53}{\textbf{#1}}}
\newcommand{\ImportTok}[1]{#1}
\newcommand{\InformationTok}[1]{\textcolor[rgb]{0.56,0.35,0.01}{\textbf{\textit{#1}}}}
\newcommand{\KeywordTok}[1]{\textcolor[rgb]{0.13,0.29,0.53}{\textbf{#1}}}
\newcommand{\NormalTok}[1]{#1}
\newcommand{\OperatorTok}[1]{\textcolor[rgb]{0.81,0.36,0.00}{\textbf{#1}}}
\newcommand{\OtherTok}[1]{\textcolor[rgb]{0.56,0.35,0.01}{#1}}
\newcommand{\PreprocessorTok}[1]{\textcolor[rgb]{0.56,0.35,0.01}{\textit{#1}}}
\newcommand{\RegionMarkerTok}[1]{#1}
\newcommand{\SpecialCharTok}[1]{\textcolor[rgb]{0.81,0.36,0.00}{\textbf{#1}}}
\newcommand{\SpecialStringTok}[1]{\textcolor[rgb]{0.31,0.60,0.02}{#1}}
\newcommand{\StringTok}[1]{\textcolor[rgb]{0.31,0.60,0.02}{#1}}
\newcommand{\VariableTok}[1]{\textcolor[rgb]{0.00,0.00,0.00}{#1}}
\newcommand{\VerbatimStringTok}[1]{\textcolor[rgb]{0.31,0.60,0.02}{#1}}
\newcommand{\WarningTok}[1]{\textcolor[rgb]{0.56,0.35,0.01}{\textbf{\textit{#1}}}}
\usepackage{graphicx}
\makeatletter
\def\maxwidth{\ifdim\Gin@nat@width>\linewidth\linewidth\else\Gin@nat@width\fi}
\def\maxheight{\ifdim\Gin@nat@height>\textheight\textheight\else\Gin@nat@height\fi}
\makeatother
% Scale images if necessary, so that they will not overflow the page
% margins by default, and it is still possible to overwrite the defaults
% using explicit options in \includegraphics[width, height, ...]{}
\setkeys{Gin}{width=\maxwidth,height=\maxheight,keepaspectratio}
% Set default figure placement to htbp
\makeatletter
\def\fps@figure{htbp}
\makeatother
\setlength{\emergencystretch}{3em} % prevent overfull lines
\providecommand{\tightlist}{%
  \setlength{\itemsep}{0pt}\setlength{\parskip}{0pt}}
\setcounter{secnumdepth}{-\maxdimen} % remove section numbering
\ifLuaTeX
  \usepackage{selnolig}  % disable illegal ligatures
\fi
\IfFileExists{bookmark.sty}{\usepackage{bookmark}}{\usepackage{hyperref}}
\IfFileExists{xurl.sty}{\usepackage{xurl}}{} % add URL line breaks if available
\urlstyle{same}
\hypersetup{
  pdftitle={ORIE5550 - HW1 - lc2234},
  pdfauthor={Luis Alonso Cendra Villalobos (lc2234)},
  hidelinks,
  pdfcreator={LaTeX via pandoc}}

\title{ORIE5550 - HW1 - lc2234}
\author{Luis Alonso Cendra Villalobos (lc2234)}
\date{2024-02-01}

\begin{document}
\maketitle

install.packages(``tinytex'')

\hypertarget{problem-1}{%
\section{Problem 1}\label{problem-1}}

\hypertarget{find-2-univariate-time-series-in-different-fields-online-and-do-the-following-for-each-of-the-two-series}{%
\subsection{Find 2 univariate time series in different fields online and
do the following for each of the two
series:}\label{find-2-univariate-time-series-in-different-fields-online-and-do-the-following-for-each-of-the-two-series}}

\hypertarget{a-indicate-the-exact-source-for-the-time-series-data.}{%
\subsubsection{(a) Indicate the exact source for the time series
data.}\label{a-indicate-the-exact-source-for-the-time-series-data.}}

Source of data1: World Bank. URL:
\url{https://data.worldbank.org/indicator/SE.PRM.CMPT.FE.ZS?locations=1W\&start=1973\&view=chart}

Source of data2: Yahoo! Finance. URL:
\url{https://finance.yahoo.com/quote/AAPL/history?period1=1548720000\&period2=1706486400\&interval=1mo\&filter=history\&frequency=1mo\&includeAdjustedClose=true}

\hypertarget{b-output-the-first-20-elements-x1x20-of-the-series-in-r.}{%
\subsubsection{(b) Output the first 20 elements x1,\ldots,x20 of the
series in
R.}\label{b-output-the-first-20-elements-x1x20-of-the-series-in-r.}}

\begin{Shaded}
\begin{Highlighting}[]
\CommentTok{\#Set working directory}
\FunctionTok{setwd}\NormalTok{(}\StringTok{"/Users/alons/OneDrive {-} Cornell University/Cornell University/Spring 2024/ORIE 5550/ORIE5550\_Homework1"}\NormalTok{)}
\CommentTok{\#extract data from CSV}
\NormalTok{data1 }\OtherTok{\textless{}{-}} \FunctionTok{read.csv}\NormalTok{(}\StringTok{"ORIE5550\_DataHW1.csv"}\NormalTok{)}
\NormalTok{data1}\SpecialCharTok{$}\NormalTok{Year }\OtherTok{\textless{}{-}} \FunctionTok{as.Date}\NormalTok{(data1}\SpecialCharTok{$}\NormalTok{Year)}
\NormalTok{data2 }\OtherTok{\textless{}{-}} \FunctionTok{read.csv}\NormalTok{(}\StringTok{"ORIE5550\_Data2HW1.csv"}\NormalTok{)}
\NormalTok{data2}\SpecialCharTok{$}\NormalTok{Date }\OtherTok{\textless{}{-}} \FunctionTok{as.Date}\NormalTok{(data2}\SpecialCharTok{$}\NormalTok{Date)}


\CommentTok{\#transform data to Time Series}
\NormalTok{time\_series\_data1 }\OtherTok{\textless{}{-}} \FunctionTok{ts}\NormalTok{(data1}\SpecialCharTok{$}\NormalTok{Value, }\AttributeTok{start =} \FunctionTok{min}\NormalTok{(data1}\SpecialCharTok{$}\NormalTok{Year), }\AttributeTok{frequency =} \DecValTok{1}\NormalTok{)}
\NormalTok{time\_series\_data2 }\OtherTok{\textless{}{-}} \FunctionTok{ts}\NormalTok{(data2}\SpecialCharTok{$}\NormalTok{Value, }\AttributeTok{start =} \FunctionTok{min}\NormalTok{(data2}\SpecialCharTok{$}\NormalTok{Date), }\AttributeTok{frequency =} \DecValTok{12}\NormalTok{)}

\CommentTok{\# Output the first 20 values}
\NormalTok{first\_20\_values1 }\OtherTok{\textless{}{-}} \FunctionTok{head}\NormalTok{(time\_series\_data1, }\DecValTok{20}\NormalTok{)}
\FunctionTok{print}\NormalTok{(first\_20\_values1)}
\end{Highlighting}
\end{Shaded}

\begin{verbatim}
##  [1] 70.67788 69.07636 68.12719 68.04817 71.75885 73.93382 73.02761 73.40257
##  [9] 73.72074 73.15771 75.41759 75.92915 76.64542 76.75523 77.51466 77.79708
## [17] 77.87164 77.33327 77.51179 76.97076
\end{verbatim}

\begin{Shaded}
\begin{Highlighting}[]
\NormalTok{first\_20\_values2 }\OtherTok{\textless{}{-}} \FunctionTok{head}\NormalTok{(time\_series\_data2, }\DecValTok{20}\NormalTok{)}
\FunctionTok{print}\NormalTok{(first\_20\_values2)}
\end{Highlighting}
\end{Shaded}

\begin{verbatim}
##  [1]  41.54781  45.77450  48.35781  42.18869  47.87880  51.53647  50.49627
##  [8]  54.38640  60.40613  64.89605  71.52080  75.38364  66.57903  62.08135
## [15]  71.72715  77.62061  89.30191 104.04848 126.35438 113.60417
\end{verbatim}

\hypertarget{c-produce-a-time-plot-of-the-series-in-r-after-transforming-the-series-into-a-time-series-object.}{%
\subsubsection{(c) Produce a time plot of the series in R, after
transforming the series into a time series
object.}\label{c-produce-a-time-plot-of-the-series-in-r-after-transforming-the-series-into-a-time-series-object.}}

\begin{Shaded}
\begin{Highlighting}[]
\FunctionTok{par}\NormalTok{(}\AttributeTok{mfrow=}\FunctionTok{c}\NormalTok{(}\DecValTok{2}\NormalTok{,}\DecValTok{1}\NormalTok{))}
\FunctionTok{plot.ts}\NormalTok{(time\_series\_data1, }\AttributeTok{ylab=}\StringTok{\textquotesingle{}Percentage(\%)\textquotesingle{}}\NormalTok{, }\AttributeTok{xlab=}\StringTok{\textquotesingle{}Year\textquotesingle{}}\NormalTok{, }
        \AttributeTok{main =} \StringTok{"Percentage females with complete primary education (World), 1970 {-} 2020"}\NormalTok{,}
        \AttributeTok{sub =} \StringTok{"Source: World Bank"}\NormalTok{)}


\FunctionTok{plot.ts}\NormalTok{(time\_series\_data2, }\AttributeTok{ylab=}\StringTok{\textquotesingle{}Price\textquotesingle{}}\NormalTok{, }\AttributeTok{xlab=}\StringTok{\textquotesingle{}Date\textquotesingle{}}\NormalTok{,}
        \AttributeTok{main =} \StringTok{"Apple stock price, Jan. 2019 {-} Dec. 2023"}\NormalTok{, }
        \AttributeTok{sub =} \StringTok{"Source: Yahoo! Finance"}\NormalTok{)}
\end{Highlighting}
\end{Shaded}

\begin{center}\includegraphics{ORIE5550_HW1_lc2234_Markdown_files/figure-latex/unnamed-chunk-2-1} \end{center}

\hypertarget{d-discuss-briefly-possible-objectives-for-analyzing-the-time-series.}{%
\subsubsection{(d) Discuss briefly possible objectives for analyzing the
time
series.}\label{d-discuss-briefly-possible-objectives-for-analyzing-the-time-series.}}

The percentage of females with complete primary education can be
analyzed to study the factors that are involved in the literacy rate of
women worldwide, we can regress the series against others like
investment in education, inclusion of women to the education system,
gender equality policies/decisions, among others. The Apple stock price
can be examined to determine factors that can forecast the stock price
in the future with the objective of making investment decisions, for
example.

\hypertarget{problem-2}{%
\section{Problem 2}\label{problem-2}}

\hypertarget{a-produce-two-different-realizations-xt-t-1t-of-the-model-of-length-t-100-include-the-r-code}{%
\subsubsection{(a) Produce two different realizations xt, t =
1,\ldots,T, of the model of length T = 100; Include the R
code;}\label{a-produce-two-different-realizations-xt-t-1t-of-the-model-of-length-t-100-include-the-r-code}}

\begin{Shaded}
\begin{Highlighting}[]
\CommentTok{\# Set the length of the time series}
\NormalTok{length\_of\_series }\OtherTok{\textless{}{-}} \DecValTok{100}

\CommentTok{\# Initialize vectors to store the time series values}
\NormalTok{X\_1 }\OtherTok{\textless{}{-}} \FunctionTok{numeric}\NormalTok{(length\_of\_series)}
\NormalTok{Y\_1 }\OtherTok{\textless{}{-}} \FunctionTok{numeric}\NormalTok{(length\_of\_series)}
\NormalTok{X\_2 }\OtherTok{\textless{}{-}} \FunctionTok{numeric}\NormalTok{(length\_of\_series)}
\NormalTok{Y\_2 }\OtherTok{\textless{}{-}} \FunctionTok{numeric}\NormalTok{(length\_of\_series)}
\NormalTok{Z\_1 }\OtherTok{\textless{}{-}} \FunctionTok{numeric}\NormalTok{(length\_of\_series)}
\NormalTok{Z\_2 }\OtherTok{\textless{}{-}} \FunctionTok{numeric}\NormalTok{(length\_of\_series)}


\CommentTok{\# Set the initial value of Z}
\NormalTok{Z\_1[}\DecValTok{1}\NormalTok{] }\OtherTok{\textless{}{-}} \FunctionTok{sample}\NormalTok{(}\FunctionTok{c}\NormalTok{(}\DecValTok{2}\NormalTok{, }\SpecialCharTok{{-}}\DecValTok{1}\NormalTok{), }\AttributeTok{size =} \DecValTok{1}\NormalTok{, }\AttributeTok{prob =} \FunctionTok{c}\NormalTok{(}\DecValTok{1}\SpecialCharTok{/}\DecValTok{3}\NormalTok{, }\DecValTok{2}\SpecialCharTok{/}\DecValTok{3}\NormalTok{)) }\CommentTok{\#realization 1 of Z[1]}
\NormalTok{Z\_2[}\DecValTok{1}\NormalTok{] }\OtherTok{\textless{}{-}} \FunctionTok{sample}\NormalTok{(}\FunctionTok{c}\NormalTok{(}\DecValTok{2}\NormalTok{, }\SpecialCharTok{{-}}\DecValTok{1}\NormalTok{), }\AttributeTok{size =} \DecValTok{1}\NormalTok{, }\AttributeTok{prob =} \FunctionTok{c}\NormalTok{(}\DecValTok{1}\SpecialCharTok{/}\DecValTok{3}\NormalTok{, }\DecValTok{2}\SpecialCharTok{/}\DecValTok{3}\NormalTok{)) }\CommentTok{\#realization 2 of Z[1]}

\CommentTok{\# Generate the time series realizations for Z\_t, X\_t, and X\_t * X\_(t{-}1)}
\ControlFlowTok{for}\NormalTok{ (t }\ControlFlowTok{in} \DecValTok{2}\SpecialCharTok{:}\NormalTok{length\_of\_series) \{}
\NormalTok{  Z\_1[t] }\OtherTok{\textless{}{-}} \FunctionTok{sample}\NormalTok{(}\FunctionTok{c}\NormalTok{(}\DecValTok{2}\NormalTok{, }\SpecialCharTok{{-}}\DecValTok{1}\NormalTok{), }\AttributeTok{size =} \DecValTok{1}\NormalTok{, }\AttributeTok{prob =} \FunctionTok{c}\NormalTok{(}\DecValTok{1}\SpecialCharTok{/}\DecValTok{3}\NormalTok{, }\DecValTok{2}\SpecialCharTok{/}\DecValTok{3}\NormalTok{))}
\NormalTok{  Z\_2[t] }\OtherTok{\textless{}{-}} \FunctionTok{sample}\NormalTok{(}\FunctionTok{c}\NormalTok{(}\DecValTok{2}\NormalTok{, }\SpecialCharTok{{-}}\DecValTok{1}\NormalTok{), }\AttributeTok{size =} \DecValTok{1}\NormalTok{, }\AttributeTok{prob =} \FunctionTok{c}\NormalTok{(}\DecValTok{1}\SpecialCharTok{/}\DecValTok{3}\NormalTok{, }\DecValTok{2}\SpecialCharTok{/}\DecValTok{3}\NormalTok{))}
  
  \CommentTok{\# Realization 1 of X\_t and Y = X\_t * X\_(t{-}1)}
\NormalTok{  X\_1[t] }\OtherTok{\textless{}{-}} \DecValTok{2} \SpecialCharTok{*}\NormalTok{ Z\_1[t] }\SpecialCharTok{+}\NormalTok{ Z\_1[t }\SpecialCharTok{{-}} \DecValTok{1}\NormalTok{]}
\NormalTok{  Y\_1[t] }\OtherTok{\textless{}{-}}\NormalTok{ X\_1[t] }\SpecialCharTok{*}\NormalTok{ X\_1[t }\SpecialCharTok{{-}} \DecValTok{1}\NormalTok{]}
  
  \CommentTok{\# Realization 2 of X\_t and Y = X\_t * X\_(t{-}1)}
\NormalTok{  X\_2[t] }\OtherTok{\textless{}{-}} \DecValTok{2} \SpecialCharTok{*}\NormalTok{ Z\_2[t] }\SpecialCharTok{+}\NormalTok{ Z\_2[t }\SpecialCharTok{{-}} \DecValTok{1}\NormalTok{]}
\NormalTok{  Y\_2[t] }\OtherTok{\textless{}{-}}\NormalTok{ X\_2[t] }\SpecialCharTok{*}\NormalTok{ X\_2[t }\SpecialCharTok{{-}} \DecValTok{1}\NormalTok{]}
  
\NormalTok{\}}
  
\CommentTok{\# Calculate the mean of Z}
\NormalTok{mean\_Z\_1 }\OtherTok{\textless{}{-}} \FunctionTok{mean}\NormalTok{(Z\_1) }\CommentTok{\#mean of realization 1}
\NormalTok{mean\_Z\_2 }\OtherTok{\textless{}{-}} \FunctionTok{mean}\NormalTok{(Z\_2) }\CommentTok{\#mean of realization 2}

\CommentTok{\# Calculate the mean of Z\^{}2}
\NormalTok{mean\_Z\_1\_squared }\OtherTok{\textless{}{-}} \FunctionTok{mean}\NormalTok{(Z\_1}\SpecialCharTok{\^{}}\DecValTok{2}\NormalTok{) }\CommentTok{\#mean of realization 1 squared}
\NormalTok{mean\_Z\_2\_squared }\OtherTok{\textless{}{-}} \FunctionTok{mean}\NormalTok{(Z\_2}\SpecialCharTok{\^{}}\DecValTok{2}\NormalTok{) }\CommentTok{\#mean of realization 2 squared}

\CommentTok{\# Display the results}
\FunctionTok{print}\NormalTok{(}\FunctionTok{paste}\NormalTok{(}\StringTok{"Mean of Z\_1:"}\NormalTok{, mean\_Z\_1))}
\end{Highlighting}
\end{Shaded}

\begin{verbatim}
## [1] "Mean of Z_1: 0.02"
\end{verbatim}

\begin{Shaded}
\begin{Highlighting}[]
\FunctionTok{print}\NormalTok{(}\FunctionTok{paste}\NormalTok{(}\StringTok{"Mean of Z\_1\^{}2:"}\NormalTok{, mean\_Z\_1\_squared))}
\end{Highlighting}
\end{Shaded}

\begin{verbatim}
## [1] "Mean of Z_1^2: 2.02"
\end{verbatim}

\begin{Shaded}
\begin{Highlighting}[]
\FunctionTok{print}\NormalTok{(}\FunctionTok{paste}\NormalTok{(}\StringTok{"Mean of Z\_1:"}\NormalTok{, mean\_Z\_2))}
\end{Highlighting}
\end{Shaded}

\begin{verbatim}
## [1] "Mean of Z_1: -0.28"
\end{verbatim}

\begin{Shaded}
\begin{Highlighting}[]
\FunctionTok{print}\NormalTok{(}\FunctionTok{paste}\NormalTok{(}\StringTok{"Mean of Z\_1\^{}2:"}\NormalTok{, mean\_Z\_2\_squared))}
\end{Highlighting}
\end{Shaded}

\begin{verbatim}
## [1] "Mean of Z_1^2: 1.72"
\end{verbatim}

\hypertarget{b-compute-theoretically-ext-and-ex2-t-compare-these-quantities-with-the-two-realizations-above-include-the-r-code.}{%
\subsubsection{(b) Compute theoretically EXt and E(X2 t); Compare these
quantities with the two realizations above; Include the R
code.}\label{b-compute-theoretically-ext-and-ex2-t-compare-these-quantities-with-the-two-realizations-above-include-the-r-code.}}

\begin{Shaded}
\begin{Highlighting}[]
\CommentTok{\# Calculate the mean of X\_t}
\NormalTok{mean\_X\_1 }\OtherTok{\textless{}{-}} \FunctionTok{mean}\NormalTok{(X\_1)}
\NormalTok{mean\_X\_2 }\OtherTok{\textless{}{-}} \FunctionTok{mean}\NormalTok{(X\_2)}

\CommentTok{\# Display the results}
\FunctionTok{print}\NormalTok{(}\FunctionTok{paste}\NormalTok{(}\StringTok{"Mean of X\_1:"}\NormalTok{, mean\_X\_1))}
\end{Highlighting}
\end{Shaded}

\begin{verbatim}
## [1] "Mean of X_1: 0.09"
\end{verbatim}

\begin{Shaded}
\begin{Highlighting}[]
\FunctionTok{print}\NormalTok{(}\FunctionTok{paste}\NormalTok{(}\StringTok{"Mean of X\_2:"}\NormalTok{, mean\_X\_2))}
\end{Highlighting}
\end{Shaded}

\begin{verbatim}
## [1] "Mean of X_2: -0.81"
\end{verbatim}

\begin{Shaded}
\begin{Highlighting}[]
\CommentTok{\# Calculate the mean of X\_t squared}
\NormalTok{mean\_X\_1\_squared }\OtherTok{\textless{}{-}} \FunctionTok{mean}\NormalTok{(X\_1}\SpecialCharTok{\^{}}\DecValTok{2}\NormalTok{)}
\NormalTok{mean\_X\_2\_squared }\OtherTok{\textless{}{-}} \FunctionTok{mean}\NormalTok{(X\_2}\SpecialCharTok{\^{}}\DecValTok{2}\NormalTok{)}

\CommentTok{\# Display the results}
\FunctionTok{print}\NormalTok{(}\FunctionTok{paste}\NormalTok{(}\StringTok{"Mean of X\_1\^{}2:"}\NormalTok{, mean\_X\_1\_squared))}
\end{Highlighting}
\end{Shaded}

\begin{verbatim}
## [1] "Mean of X_1^2: 8.73"
\end{verbatim}

\begin{Shaded}
\begin{Highlighting}[]
\FunctionTok{print}\NormalTok{(}\FunctionTok{paste}\NormalTok{(}\StringTok{"Mean of X\_2\^{}2:"}\NormalTok{, mean\_X\_2\_squared))}
\end{Highlighting}
\end{Shaded}

\begin{verbatim}
## [1] "Mean of X_2^2: 8.55"
\end{verbatim}

\begin{Shaded}
\begin{Highlighting}[]
\CommentTok{\# Calculate the mean of Y = X\_t * X\_(t{-}1)}
\NormalTok{mean\_Y\_1 }\OtherTok{\textless{}{-}} \FunctionTok{mean}\NormalTok{(Y\_1)}
\NormalTok{mean\_Y\_2 }\OtherTok{\textless{}{-}} \FunctionTok{mean}\NormalTok{(Y\_2)}

\CommentTok{\# Display the results}
\FunctionTok{print}\NormalTok{(}\StringTok{"Let Y = X\_t * X\_(t{-}1), then:"}\NormalTok{)}
\end{Highlighting}
\end{Shaded}

\begin{verbatim}
## [1] "Let Y = X_t * X_(t-1), then:"
\end{verbatim}

\begin{Shaded}
\begin{Highlighting}[]
\FunctionTok{print}\NormalTok{(}\FunctionTok{paste}\NormalTok{(}\StringTok{"Mean of Y\_1:"}\NormalTok{, mean\_Y\_1))}
\end{Highlighting}
\end{Shaded}

\begin{verbatim}
## [1] "Mean of Y_1: 2.16"
\end{verbatim}

\begin{Shaded}
\begin{Highlighting}[]
\FunctionTok{print}\NormalTok{(}\FunctionTok{paste}\NormalTok{(}\StringTok{"Mean of Y\_2:"}\NormalTok{, mean\_Y\_2))}
\end{Highlighting}
\end{Shaded}

\begin{verbatim}
## [1] "Mean of Y_2: 3.69"
\end{verbatim}

\begin{Shaded}
\begin{Highlighting}[]
\CommentTok{\# Correlations}

\NormalTok{V\_11 }\OtherTok{\textless{}{-}}\NormalTok{ X\_1[}\DecValTok{2}\SpecialCharTok{:}\NormalTok{length\_of\_series]}
\NormalTok{V\_12 }\OtherTok{\textless{}{-}}\NormalTok{ X\_1[}\DecValTok{1}\SpecialCharTok{:}\NormalTok{(length\_of\_series }\SpecialCharTok{{-}} \DecValTok{1}\NormalTok{)]}

\NormalTok{V\_21 }\OtherTok{\textless{}{-}}\NormalTok{ X\_2[}\DecValTok{2}\SpecialCharTok{:}\NormalTok{length\_of\_series]}
\NormalTok{V\_22 }\OtherTok{\textless{}{-}}\NormalTok{ X\_2[}\DecValTok{1}\SpecialCharTok{:}\NormalTok{(length\_of\_series }\SpecialCharTok{{-}} \DecValTok{1}\NormalTok{)]}

\CommentTok{\# Calculate the correlation between V\_1 and V\_2}
\NormalTok{correlation\_V1\_V2\_1 }\OtherTok{\textless{}{-}} \FunctionTok{cor}\NormalTok{(V\_11, V\_12)}
\NormalTok{correlation\_V1\_V2\_2 }\OtherTok{\textless{}{-}} \FunctionTok{cor}\NormalTok{(V\_21, V\_22)}

\CommentTok{\# Display the correlation}
\FunctionTok{print}\NormalTok{(}\FunctionTok{paste}\NormalTok{(}\StringTok{"Correlation between V\_1 and V\_2 realization 1:"}\NormalTok{, correlation\_V1\_V2\_1))}
\end{Highlighting}
\end{Shaded}

\begin{verbatim}
## [1] "Correlation between V_1 and V_2 realization 1: 0.247776593090737"
\end{verbatim}

\begin{Shaded}
\begin{Highlighting}[]
\FunctionTok{print}\NormalTok{(}\FunctionTok{paste}\NormalTok{(}\StringTok{"Correlation between V\_1 and V\_2 realization 2:"}\NormalTok{, correlation\_V1\_V2\_2))}
\end{Highlighting}
\end{Shaded}

\begin{verbatim}
## [1] "Correlation between V_1 and V_2 realization 2: 0.387959304165155"
\end{verbatim}

\hypertarget{d-what-are-the-possible-values-of-xt-and-the-probabilities-that-xt-takes-these-values-compare-your-answers-from-the-two-realizations-above}{%
\subsubsection{(d) What are the possible values of Xt and the
probabilities that Xt takes these values? Compare your answers from the
two realizations
above}\label{d-what-are-the-possible-values-of-xt-and-the-probabilities-that-xt-takes-these-values-compare-your-answers-from-the-two-realizations-above}}

See handwritten notes.

\hypertarget{problem-3}{%
\section{Problem 3}\label{problem-3}}

\hypertarget{problem-3.-come-up-with-your-own-time-series-model-and-repeat-parts-a-c-of-problem-2-for-the-model.-your-model-should-have-at-least-a-trend-or-a-periodic-component-and-incorporate-iid-noise-in-some-way}{%
\subsubsection{Problem 3. Come up with your own time series model and
repeat parts (a)-(c) of Problem 2 for the model. Your model should have
at least a trend or a periodic component, and incorporate IID noise in
some
way}\label{problem-3.-come-up-with-your-own-time-series-model-and-repeat-parts-a-c-of-problem-2-for-the-model.-your-model-should-have-at-least-a-trend-or-a-periodic-component-and-incorporate-iid-noise-in-some-way}}

See Handwritten notes. We assume T=100, alpha = 2, variance = 2.

\begin{Shaded}
\begin{Highlighting}[]
\CommentTok{\# Set the length of the time series}
\NormalTok{length\_of\_series }\OtherTok{\textless{}{-}} \DecValTok{100}

\CommentTok{\# Initialize vectors to store the time series values}
\NormalTok{X\_1 }\OtherTok{\textless{}{-}} \FunctionTok{numeric}\NormalTok{(length\_of\_series)}
\NormalTok{Y\_1 }\OtherTok{\textless{}{-}} \FunctionTok{numeric}\NormalTok{(length\_of\_series)}
\NormalTok{X\_2 }\OtherTok{\textless{}{-}} \FunctionTok{numeric}\NormalTok{(length\_of\_series)}
\NormalTok{Y\_2 }\OtherTok{\textless{}{-}} \FunctionTok{numeric}\NormalTok{(length\_of\_series)}
\NormalTok{Z\_1 }\OtherTok{\textless{}{-}} \FunctionTok{numeric}\NormalTok{(length\_of\_series)}
\NormalTok{Z\_2 }\OtherTok{\textless{}{-}} \FunctionTok{numeric}\NormalTok{(length\_of\_series)}


\CommentTok{\# Create a time series Z\_t with elements distributed normally}
\NormalTok{Z\_1 }\OtherTok{\textless{}{-}} \FunctionTok{rnorm}\NormalTok{(length\_of\_series, }\AttributeTok{mean =} \DecValTok{0}\NormalTok{, }\AttributeTok{sd =} \FunctionTok{sqrt}\NormalTok{(}\DecValTok{2}\NormalTok{))}
\NormalTok{Z\_2 }\OtherTok{\textless{}{-}} \FunctionTok{rnorm}\NormalTok{(length\_of\_series, }\AttributeTok{mean =} \DecValTok{0}\NormalTok{, }\AttributeTok{sd =} \FunctionTok{sqrt}\NormalTok{(}\DecValTok{2}\NormalTok{))}

\NormalTok{X\_1[}\DecValTok{1}\NormalTok{] }\OtherTok{\textless{}{-}} \DecValTok{2} \SpecialCharTok{+}\NormalTok{ Z\_1[}\DecValTok{1}\NormalTok{]}
\NormalTok{X\_2[}\DecValTok{1}\NormalTok{] }\OtherTok{\textless{}{-}} \DecValTok{2} \SpecialCharTok{+}\NormalTok{ Z\_2[}\DecValTok{1}\NormalTok{]}

\CommentTok{\# Generate the time series realizations for Z\_t, X\_t, and X\_t * X\_(t{-}1)}
\ControlFlowTok{for}\NormalTok{ (t }\ControlFlowTok{in} \DecValTok{2}\SpecialCharTok{:}\NormalTok{length\_of\_series) \{}
  
  \CommentTok{\# Realization 1 of X\_t and Y = X\_t * X\_(t{-}1)}
\NormalTok{  X\_1[t] }\OtherTok{\textless{}{-}} \DecValTok{2} \SpecialCharTok{*}\NormalTok{ t }\SpecialCharTok{+}\NormalTok{ Z\_1[t]}
\NormalTok{  Y\_1[t] }\OtherTok{\textless{}{-}}\NormalTok{ X\_1[t] }\SpecialCharTok{*}\NormalTok{ X\_1[t }\SpecialCharTok{{-}} \DecValTok{1}\NormalTok{]}
  
  \CommentTok{\# Realization 2 of X\_t and Y = X\_t * X\_(t{-}1)}
\NormalTok{  X\_2[t] }\OtherTok{\textless{}{-}} \DecValTok{2} \SpecialCharTok{*}\NormalTok{ t }\SpecialCharTok{+}\NormalTok{ Z\_2[t]}
\NormalTok{  Y\_2[t] }\OtherTok{\textless{}{-}}\NormalTok{ X\_2[t] }\SpecialCharTok{*}\NormalTok{ X\_2[t }\SpecialCharTok{{-}} \DecValTok{1}\NormalTok{]}
  
\NormalTok{\}}
  
\CommentTok{\# Calculate the mean of Z}
\NormalTok{mean\_Z\_1 }\OtherTok{\textless{}{-}} \FunctionTok{mean}\NormalTok{(Z\_1) }\CommentTok{\#mean of realization 1}
\NormalTok{mean\_Z\_2 }\OtherTok{\textless{}{-}} \FunctionTok{mean}\NormalTok{(Z\_2) }\CommentTok{\#mean of realization 2}

\CommentTok{\# Calculate the mean of Z\^{}2}
\NormalTok{mean\_Z\_1\_squared }\OtherTok{\textless{}{-}} \FunctionTok{mean}\NormalTok{(Z\_1}\SpecialCharTok{\^{}}\DecValTok{2}\NormalTok{) }\CommentTok{\#mean of realization 1 squared}
\NormalTok{mean\_Z\_2\_squared }\OtherTok{\textless{}{-}} \FunctionTok{mean}\NormalTok{(Z\_2}\SpecialCharTok{\^{}}\DecValTok{2}\NormalTok{) }\CommentTok{\#mean of realization 2 squared}

\CommentTok{\# Display the results}
\FunctionTok{print}\NormalTok{(}\FunctionTok{paste}\NormalTok{(}\StringTok{"Mean of Z\_1:"}\NormalTok{, mean\_Z\_1))}
\end{Highlighting}
\end{Shaded}

\begin{verbatim}
## [1] "Mean of Z_1: 0.0872531081324089"
\end{verbatim}

\begin{Shaded}
\begin{Highlighting}[]
\FunctionTok{print}\NormalTok{(}\FunctionTok{paste}\NormalTok{(}\StringTok{"Mean of Z\_1\^{}2:"}\NormalTok{, mean\_Z\_1\_squared))}
\end{Highlighting}
\end{Shaded}

\begin{verbatim}
## [1] "Mean of Z_1^2: 1.90780846047563"
\end{verbatim}

\begin{Shaded}
\begin{Highlighting}[]
\FunctionTok{print}\NormalTok{(}\FunctionTok{paste}\NormalTok{(}\StringTok{"Mean of Z\_1:"}\NormalTok{, mean\_Z\_2))}
\end{Highlighting}
\end{Shaded}

\begin{verbatim}
## [1] "Mean of Z_1: 0.0975031026701417"
\end{verbatim}

\begin{Shaded}
\begin{Highlighting}[]
\FunctionTok{print}\NormalTok{(}\FunctionTok{paste}\NormalTok{(}\StringTok{"Mean of Z\_1\^{}2:"}\NormalTok{, mean\_Z\_2\_squared))}
\end{Highlighting}
\end{Shaded}

\begin{verbatim}
## [1] "Mean of Z_1^2: 2.14682783358742"
\end{verbatim}

\begin{Shaded}
\begin{Highlighting}[]
\CommentTok{\# Calculate the mean of X\_t}
\NormalTok{mean\_X\_1 }\OtherTok{\textless{}{-}} \FunctionTok{mean}\NormalTok{(X\_1)}
\NormalTok{mean\_X\_2 }\OtherTok{\textless{}{-}} \FunctionTok{mean}\NormalTok{(X\_2)}

\CommentTok{\# Display the results}
\FunctionTok{print}\NormalTok{(}\FunctionTok{paste}\NormalTok{(}\StringTok{"Mean of X\_1:"}\NormalTok{, mean\_X\_1))}
\end{Highlighting}
\end{Shaded}

\begin{verbatim}
## [1] "Mean of X_1: 101.087253108132"
\end{verbatim}

\begin{Shaded}
\begin{Highlighting}[]
\FunctionTok{print}\NormalTok{(}\FunctionTok{paste}\NormalTok{(}\StringTok{"Mean of X\_2:"}\NormalTok{, mean\_X\_2))}
\end{Highlighting}
\end{Shaded}

\begin{verbatim}
## [1] "Mean of X_2: 101.09750310267"
\end{verbatim}

\begin{Shaded}
\begin{Highlighting}[]
\CommentTok{\# Calculate the mean of X\_t squared}
\NormalTok{mean\_X\_1\_squared }\OtherTok{\textless{}{-}} \FunctionTok{mean}\NormalTok{(X\_1}\SpecialCharTok{\^{}}\DecValTok{2}\NormalTok{)}
\NormalTok{mean\_X\_2\_squared }\OtherTok{\textless{}{-}} \FunctionTok{mean}\NormalTok{(X\_2}\SpecialCharTok{\^{}}\DecValTok{2}\NormalTok{)}

\CommentTok{\# Display the results}
\FunctionTok{print}\NormalTok{(}\FunctionTok{paste}\NormalTok{(}\StringTok{"Mean of X\_1\^{}2:"}\NormalTok{, mean\_X\_1\_squared))}
\end{Highlighting}
\end{Shaded}

\begin{verbatim}
## [1] "Mean of X_1^2: 13567.9144086239"
\end{verbatim}

\begin{Shaded}
\begin{Highlighting}[]
\FunctionTok{print}\NormalTok{(}\FunctionTok{paste}\NormalTok{(}\StringTok{"Mean of X\_2\^{}2:"}\NormalTok{, mean\_X\_2\_squared))}
\end{Highlighting}
\end{Shaded}

\begin{verbatim}
## [1] "Mean of X_2^2: 13543.4513730565"
\end{verbatim}

\begin{Shaded}
\begin{Highlighting}[]
\CommentTok{\# Calculate the mean of Y = X\_t * X\_(t{-}1)}
\NormalTok{mean\_Y\_1 }\OtherTok{\textless{}{-}} \FunctionTok{mean}\NormalTok{(Y\_1)}
\NormalTok{mean\_Y\_2 }\OtherTok{\textless{}{-}} \FunctionTok{mean}\NormalTok{(Y\_2)}

\CommentTok{\# Display the results}
\FunctionTok{print}\NormalTok{(}\StringTok{"Let Y = X\_t * X\_(t{-}1), then:"}\NormalTok{)}
\end{Highlighting}
\end{Shaded}

\begin{verbatim}
## [1] "Let Y = X_t * X_(t-1), then:"
\end{verbatim}

\begin{Shaded}
\begin{Highlighting}[]
\FunctionTok{print}\NormalTok{(}\FunctionTok{paste}\NormalTok{(}\StringTok{"Mean of Y\_1:"}\NormalTok{, mean\_Y\_1))}
\end{Highlighting}
\end{Shaded}

\begin{verbatim}
## [1] "Mean of Y_1: 13367.6267622716"
\end{verbatim}

\begin{Shaded}
\begin{Highlighting}[]
\FunctionTok{print}\NormalTok{(}\FunctionTok{paste}\NormalTok{(}\StringTok{"Mean of Y\_2:"}\NormalTok{, mean\_Y\_2))}
\end{Highlighting}
\end{Shaded}

\begin{verbatim}
## [1] "Mean of Y_2: 13340.2505018336"
\end{verbatim}

\begin{Shaded}
\begin{Highlighting}[]
\CommentTok{\# Correlations}

\NormalTok{V\_11 }\OtherTok{\textless{}{-}}\NormalTok{ X\_1[}\DecValTok{2}\SpecialCharTok{:}\NormalTok{length\_of\_series]}
\NormalTok{V\_12 }\OtherTok{\textless{}{-}}\NormalTok{ X\_1[}\DecValTok{1}\SpecialCharTok{:}\NormalTok{(length\_of\_series }\SpecialCharTok{{-}} \DecValTok{1}\NormalTok{)]}

\NormalTok{V\_21 }\OtherTok{\textless{}{-}}\NormalTok{ X\_2[}\DecValTok{2}\SpecialCharTok{:}\NormalTok{length\_of\_series]}
\NormalTok{V\_22 }\OtherTok{\textless{}{-}}\NormalTok{ X\_2[}\DecValTok{1}\SpecialCharTok{:}\NormalTok{(length\_of\_series }\SpecialCharTok{{-}} \DecValTok{1}\NormalTok{)]}

\CommentTok{\# Calculate the correlation between V\_1 and V\_2}
\NormalTok{correlation\_V1\_V2\_1 }\OtherTok{\textless{}{-}} \FunctionTok{cor}\NormalTok{(V\_11, V\_12)}
\NormalTok{correlation\_V1\_V2\_2 }\OtherTok{\textless{}{-}} \FunctionTok{cor}\NormalTok{(V\_21, V\_22)}

\CommentTok{\# Display the correlation}
\FunctionTok{print}\NormalTok{(}\FunctionTok{paste}\NormalTok{(}\StringTok{"Correlation between V\_1 and V\_2 realization 1:"}\NormalTok{, correlation\_V1\_V2\_1))}
\end{Highlighting}
\end{Shaded}

\begin{verbatim}
## [1] "Correlation between V_1 and V_2 realization 1: 0.999550080856796"
\end{verbatim}

\begin{Shaded}
\begin{Highlighting}[]
\FunctionTok{print}\NormalTok{(}\FunctionTok{paste}\NormalTok{(}\StringTok{"Correlation between V\_1 and V\_2 realization 2:"}\NormalTok{, correlation\_V1\_V2\_2))}
\end{Highlighting}
\end{Shaded}

\begin{verbatim}
## [1] "Correlation between V_1 and V_2 realization 2: 0.999329664470164"
\end{verbatim}

\hypertarget{problem-4}{%
\section{Problem 4}\label{problem-4}}

\hypertarget{a}{%
\subsubsection{(a)}\label{a}}

See Handwritten notes.

\hypertarget{b-produce-two-different-realizations-xt-t-1t-of-the-model-of-length-t-100-include-the-r-code}{%
\subsubsection{(b) Produce two different realizations xt, t =
1,\ldots,T, of the model of length T = 100; Include the R
code;}\label{b-produce-two-different-realizations-xt-t-1t-of-the-model-of-length-t-100-include-the-r-code}}

\begin{Shaded}
\begin{Highlighting}[]
\CommentTok{\# Set the parameters}
\NormalTok{mean\_value }\OtherTok{\textless{}{-}} \DecValTok{1}
\NormalTok{variance\_value }\OtherTok{\textless{}{-}} \DecValTok{4}
\NormalTok{TT }\OtherTok{\textless{}{-}} \DecValTok{100}
\NormalTok{num\_simulations }\OtherTok{\textless{}{-}} \DecValTok{10000}

\CommentTok{\# Initialize array to store values at T=100 from Monte Carlo}
\NormalTok{X\_values\_at\_T\_1 }\OtherTok{\textless{}{-}} \FunctionTok{numeric}\NormalTok{(num\_simulations)}
\NormalTok{X\_values\_at\_T\_2 }\OtherTok{\textless{}{-}} \FunctionTok{numeric}\NormalTok{(num\_simulations)}
\NormalTok{Y\_values\_at\_T\_1 }\OtherTok{\textless{}{-}} \FunctionTok{numeric}\NormalTok{(num\_simulations)}
\NormalTok{Y\_values\_at\_T\_2 }\OtherTok{\textless{}{-}} \FunctionTok{numeric}\NormalTok{(num\_simulations)}

\CommentTok{\# Perform simulations}
\ControlFlowTok{for}\NormalTok{ (sim }\ControlFlowTok{in} \DecValTok{1}\SpecialCharTok{:}\NormalTok{num\_simulations) \{}
  \CommentTok{\# Generate random variable time series Z\_t white noise Normal(1,4)}
\NormalTok{  Z\_1\_t }\OtherTok{\textless{}{-}} \FunctionTok{rnorm}\NormalTok{(TT, }\AttributeTok{mean =}\NormalTok{ mean\_value, }\AttributeTok{sd =} \FunctionTok{sqrt}\NormalTok{(variance\_value))}
\NormalTok{  Z\_2\_t }\OtherTok{\textless{}{-}} \FunctionTok{rnorm}\NormalTok{(TT, }\AttributeTok{mean =}\NormalTok{ mean\_value, }\AttributeTok{sd =} \FunctionTok{sqrt}\NormalTok{(variance\_value))}
  
  \CommentTok{\# Initialize X\_t with X\_0 = 0 for the 2 realizations}
\NormalTok{  X\_1\_t }\OtherTok{\textless{}{-}} \FunctionTok{numeric}\NormalTok{(TT)}
\NormalTok{  X\_2\_t }\OtherTok{\textless{}{-}} \FunctionTok{numeric}\NormalTok{(TT)}
\NormalTok{  Y\_1 }\OtherTok{\textless{}{-}} \FunctionTok{numeric}\NormalTok{(TT)}
\NormalTok{  Y\_2 }\OtherTok{\textless{}{-}} \FunctionTok{numeric}\NormalTok{(TT)}
\NormalTok{  X\_1\_t[}\DecValTok{1}\NormalTok{] }\OtherTok{\textless{}{-}} \DecValTok{0}
\NormalTok{  X\_2\_t[}\DecValTok{1}\NormalTok{] }\OtherTok{\textless{}{-}} \DecValTok{0}
  
  \CommentTok{\# Generate AR(1) model for both realizations}
  \ControlFlowTok{for}\NormalTok{ (t }\ControlFlowTok{in} \DecValTok{2}\SpecialCharTok{:}\NormalTok{TT) \{}
\NormalTok{    X\_1\_t[t] }\OtherTok{\textless{}{-}}\NormalTok{ X\_1\_t[t}\DecValTok{{-}1}\NormalTok{] }\SpecialCharTok{+}\NormalTok{ Z\_1\_t[t]}
\NormalTok{    X\_2\_t[t] }\OtherTok{\textless{}{-}}\NormalTok{ X\_2\_t[t}\DecValTok{{-}1}\NormalTok{] }\SpecialCharTok{+}\NormalTok{ Z\_2\_t[t]}
    
    \CommentTok{\# Realizations of Y = X\_t * X\_(t{-}1)}
\NormalTok{    Y\_1[t] }\OtherTok{\textless{}{-}}\NormalTok{ X\_1\_t[t] }\SpecialCharTok{*}\NormalTok{ X\_1\_t[t }\SpecialCharTok{{-}} \DecValTok{1}\NormalTok{]}
\NormalTok{    Y\_2[t] }\OtherTok{\textless{}{-}}\NormalTok{ X\_2\_t[t] }\SpecialCharTok{*}\NormalTok{ X\_2\_t[t }\SpecialCharTok{{-}} \DecValTok{1}\NormalTok{]}
    
\NormalTok{  \}}
  \CommentTok{\# Store the last element of X\_t in values\_at\_T for both realizations}
\NormalTok{  X\_values\_at\_T\_1[sim] }\OtherTok{\textless{}{-}}\NormalTok{ X\_1\_t[TT]}
\NormalTok{  X\_values\_at\_T\_2[sim] }\OtherTok{\textless{}{-}}\NormalTok{ X\_2\_t[TT]}
\NormalTok{  Y\_values\_at\_T\_1[sim] }\OtherTok{\textless{}{-}}\NormalTok{ Y\_1[TT]}
\NormalTok{  Y\_values\_at\_T\_2[sim] }\OtherTok{\textless{}{-}}\NormalTok{ Y\_2[TT]}
\NormalTok{\}}
\end{Highlighting}
\end{Shaded}

\hypertarget{c-compute-theoretically-ext-and-ex2-t-at-t-100-write-the-r-code-to-compute-these-quantities-note-do-not-plug-in-the-number-into-the-theoretical-result.}{%
\subsubsection{(c) Compute theoretically EXt and E(X2 t) at t = 100;
Write the R code to compute these quantities (Note: Do not plug in the
number into the theoretical
result).}\label{c-compute-theoretically-ext-and-ex2-t-at-t-100-write-the-r-code-to-compute-these-quantities-note-do-not-plug-in-the-number-into-the-theoretical-result.}}

\begin{Shaded}
\begin{Highlighting}[]
\CommentTok{\# Calculate the mean of the realizations}
\NormalTok{X\_mean\_1 }\OtherTok{\textless{}{-}} \FunctionTok{mean}\NormalTok{(X\_values\_at\_T\_1) }\CommentTok{\#mean of realization 1}
\NormalTok{X\_mean\_2 }\OtherTok{\textless{}{-}} \FunctionTok{mean}\NormalTok{(X\_values\_at\_T\_2) }\CommentTok{\#mean of realization 2}

\CommentTok{\# Calculate the mean of Z\^{}2}
\NormalTok{X\_mean\_1\_squared }\OtherTok{\textless{}{-}} \FunctionTok{mean}\NormalTok{(X\_values\_at\_T\_1}\SpecialCharTok{\^{}}\DecValTok{2}\NormalTok{) }\CommentTok{\#mean of realization 1 squared}
\NormalTok{X\_mean\_2\_squared }\OtherTok{\textless{}{-}} \FunctionTok{mean}\NormalTok{(X\_values\_at\_T\_2}\SpecialCharTok{\^{}}\DecValTok{2}\NormalTok{) }\CommentTok{\#mean of realization 2 squared}

\CommentTok{\# Display the results}
\FunctionTok{print}\NormalTok{(}\FunctionTok{paste}\NormalTok{(}\StringTok{"Mean of realization 1 of X:"}\NormalTok{, X\_mean\_1))}
\end{Highlighting}
\end{Shaded}

\begin{verbatim}
## [1] "Mean of realization 1 of X: 99.0869314191766"
\end{verbatim}

\begin{Shaded}
\begin{Highlighting}[]
\FunctionTok{print}\NormalTok{(}\FunctionTok{paste}\NormalTok{(}\StringTok{"Mean of realization 1 of X squared:"}\NormalTok{, X\_mean\_1\_squared))}
\end{Highlighting}
\end{Shaded}

\begin{verbatim}
## [1] "Mean of realization 1 of X squared: 10207.9239705549"
\end{verbatim}

\begin{Shaded}
\begin{Highlighting}[]
\FunctionTok{print}\NormalTok{(}\FunctionTok{paste}\NormalTok{(}\StringTok{"Mean of realization 2 of X:"}\NormalTok{, X\_mean\_2))}
\end{Highlighting}
\end{Shaded}

\begin{verbatim}
## [1] "Mean of realization 2 of X: 99.4281070747329"
\end{verbatim}

\begin{Shaded}
\begin{Highlighting}[]
\FunctionTok{print}\NormalTok{(}\FunctionTok{paste}\NormalTok{(}\StringTok{"Mean of realization 2 of X squared:"}\NormalTok{, X\_mean\_2\_squared))}
\end{Highlighting}
\end{Shaded}

\begin{verbatim}
## [1] "Mean of realization 2 of X squared: 10290.7276896512"
\end{verbatim}

\hypertarget{d-compute-theoretically-ext-xt1-and-corrxtxt1-at-t-100-write-the-r-code-to-compute-these-quantities-note-do-not-plug-in-the-number-into-the-theoretical-result.}{%
\subsubsection{(d) Compute theoretically E(Xt * Xt−1) and Corr(Xt,Xt−1)
at t = 100; Write the R code to compute these quantities (Note: Do not
plug in the number into the theoretical
result).}\label{d-compute-theoretically-ext-xt1-and-corrxtxt1-at-t-100-write-the-r-code-to-compute-these-quantities-note-do-not-plug-in-the-number-into-the-theoretical-result.}}

\begin{Shaded}
\begin{Highlighting}[]
\CommentTok{\# Calculate the mean of Y = X\_t * X\_(t{-}1)}
\NormalTok{mean\_Y\_1 }\OtherTok{\textless{}{-}} \FunctionTok{mean}\NormalTok{(Y\_values\_at\_T\_1)}
\NormalTok{mean\_Y\_2 }\OtherTok{\textless{}{-}} \FunctionTok{mean}\NormalTok{(Y\_values\_at\_T\_1)}

\CommentTok{\# Display the results}
\FunctionTok{print}\NormalTok{(}\StringTok{"Let Y = X\_t * X\_(t{-}1), then:"}\NormalTok{)}
\end{Highlighting}
\end{Shaded}

\begin{verbatim}
## [1] "Let Y = X_t * X_(t-1), then:"
\end{verbatim}

\begin{Shaded}
\begin{Highlighting}[]
\FunctionTok{print}\NormalTok{(}\FunctionTok{paste}\NormalTok{(}\StringTok{"Mean of realization 1 of Y:"}\NormalTok{, mean\_Y\_1))}
\end{Highlighting}
\end{Shaded}

\begin{verbatim}
## [1] "Mean of realization 1 of Y: 10102.2140670102"
\end{verbatim}

\begin{Shaded}
\begin{Highlighting}[]
\FunctionTok{print}\NormalTok{(}\FunctionTok{paste}\NormalTok{(}\StringTok{"Mean of realization 2 of Y:"}\NormalTok{, mean\_Y\_2))}
\end{Highlighting}
\end{Shaded}

\begin{verbatim}
## [1] "Mean of realization 2 of Y: 10102.2140670102"
\end{verbatim}

\begin{Shaded}
\begin{Highlighting}[]
\CommentTok{\# Correlations}
\NormalTok{V\_11 }\OtherTok{\textless{}{-}}\NormalTok{ X\_values\_at\_T\_1[}\DecValTok{2}\SpecialCharTok{:}\NormalTok{length\_of\_series]}
\NormalTok{V\_12 }\OtherTok{\textless{}{-}}\NormalTok{ X\_values\_at\_T\_1[}\DecValTok{1}\SpecialCharTok{:}\NormalTok{(length\_of\_series }\SpecialCharTok{{-}} \DecValTok{1}\NormalTok{)]}

\NormalTok{V\_21 }\OtherTok{\textless{}{-}}\NormalTok{ X\_values\_at\_T\_2[}\DecValTok{2}\SpecialCharTok{:}\NormalTok{length\_of\_series]}
\NormalTok{V\_22 }\OtherTok{\textless{}{-}}\NormalTok{ X\_values\_at\_T\_2[}\DecValTok{1}\SpecialCharTok{:}\NormalTok{(length\_of\_series }\SpecialCharTok{{-}} \DecValTok{1}\NormalTok{)]}

\CommentTok{\# Calculate the correlation between V\_1 and V\_2}
\NormalTok{correlation\_V1\_V2\_1 }\OtherTok{\textless{}{-}} \FunctionTok{cor}\NormalTok{(V\_11, V\_12)}
\NormalTok{correlation\_V1\_V2\_2 }\OtherTok{\textless{}{-}} \FunctionTok{cor}\NormalTok{(V\_21, V\_22)}

\CommentTok{\# Display the correlation}
\FunctionTok{print}\NormalTok{(}\FunctionTok{paste}\NormalTok{(}\StringTok{"Correlation between V\_1 and V\_2 realization 1:"}\NormalTok{, correlation\_V1\_V2\_1))}
\end{Highlighting}
\end{Shaded}

\begin{verbatim}
## [1] "Correlation between V_1 and V_2 realization 1: -0.0301604340412445"
\end{verbatim}

\begin{Shaded}
\begin{Highlighting}[]
\FunctionTok{print}\NormalTok{(}\FunctionTok{paste}\NormalTok{(}\StringTok{"Correlation between V\_1 and V\_2 realization 2:"}\NormalTok{, correlation\_V1\_V2\_2))}
\end{Highlighting}
\end{Shaded}

\begin{verbatim}
## [1] "Correlation between V_1 and V_2 realization 2: -0.0278127439256604"
\end{verbatim}

\end{document}
